\documentclass[notheorems,aspectratio=169]{beamer}
\usetheme[compress]{Singapore}
\setbeamertemplate{footline}[frame number]
\usenavigationsymbolstemplate{}
% \usepackage{beamerthemesplit} // Activate for custom appearance

\title{Online control of lab pond setup - change control period}
\subtitle{Experiment design and results}
%\author{Martijn Goorden}
\date{\today}

%\usepackage{etex} % Resolves dim error with pgfplots
\usepackage{subcaption}
\usepackage{appendixnumberbeamer}
\usepackage{relsize}
\usepackage{array}
\usepackage{multicol}
\usepackage{tabu}
\usepackage{bm} % Bold math
\usepackage{dsfont} % provides the blackboard 1.
\usepackage{algpseudocode} % Will also load algorithmic package
\usepackage{algorithm} % Make sure that it is called with \begin{algorithm}[H]
%\usepackage{tikz}
%\usetikzlibrary{positioning}
%\usetikzlibrary{arrows}
%\usetikzlibrary{backgrounds}
%\usetikzlibrary{calc}
%\usetikzlibrary{decorations.pathmorphing}
%\usetikzlibrary{shapes}
%\usetikzlibrary{automata}
\usepackage{pgfplots} % makes plotting data easier
\pgfplotsset{compat=1.18}
\usepgfplotslibrary{dateplot}
\usepgfplotslibrary{external} % expert graphs as pdf or eps for faster building
\tikzexternalize[prefix=tikz/]% activate the externalization.

% Attach video files in pdf
\usepackage{luatex85} % Only needed if LuaLaTeX is used, can be omitted if pdfLaTeX is used.
\usepackage{attachfile}

\definecolor{bl}{RGB}{22, 66, 115}
\definecolor{lbl}{RGB}{55, 124, 223}

\colorlet{back}{white}
\tikzset{every picture/.style={>=stealth}}
\tikzset{proc/.style={circle,draw,minimum size=2em,inner sep=0pt}}
\def\a{15} % Bending degree for automaton drawings

% Correct indentation possible when lines are split in algorithm
\makeatletter
\let\OldStatex\Statex
\renewcommand{\Statex}[1][3]{%
	\setlength\@tempdima{\algorithmicindent}%
	\OldStatex\hskip\dimexpr#1\@tempdima\relax}
\makeatother

\hypersetup{pdfpagemode=UseNone} % opens pdf without any toolbar open

%\usepackage{showframe} % show frame boundaries

\begin{document}
%
\begin{frame}
	\titlepage
\end{frame}
%

\begin{frame}{Experiment design}
	\small
	\begin{itemize}
		\item Online control: i.e., a strategy is synthesized periodically where the model is re-calibrated to the latest water level sensor reading.
		\item Experiment duration: 120 minutes.
		\item Rainfall data: first 120 minutes of the data.
		\item Initial water level: 700 mm.
		\item Physical water limit of setup: 850 mm.
		\item Duration single control period: \textbf{20 minutes} (used to be 10 minutes).
		\item Optimization cost function: $\min\  \mathbb{E}(o)$, where $o$ is the accumulated overflow duration.
		\item Fixed outflow is setting 2 (approx.\ 50\% of pump capacity).
		\item Learning budget parameters: --good-runs 100 --total-runs 200 --runs-pr-state 100 --eval-runs 100 
		\item Discretization: 0.03.
	\end{itemize}
\end{frame}

%\begin{frame}{Simulation results}
%	\centering
%	\begin{tikzpicture}
%		% To plot figure with multiple ordinates, see section 4.9.11 of the pgfplot manual.
%		pgfplotsset{set layers}
%		\begin{axis}[
%			scale only axis,
%			xlabel={Time [min]},
%			ylabel={Water level [mm]},
%			xmin=0, xmax=120,
%			ymin=600, ymax=1000,
%			axis y line*=left,  %'*' avoids arrow heads
%			axis x line*=bottom, %'*' avoids arrow heads
%			width=12cm,
%			height=5cm,
%			%			xtick distance=360,
%			xmajorgrids=true,
%			ymajorgrids=true,
%			grid style=dashed,
%			yticklabel style={
%				/pgf/number format/precision=0,
%				/pgf/number format/fixed,
%				/pgf/number format/fixed zerofill,
%			},
%			legend style={
%				at={(0.99,0.03)},
%				anchor=south east,
%				nodes={scale=0.7, transform shape}},
%			]	
%			
%			\addplot[
%			color=blue,
%			]
%			table[x expr=\thisrowno{0},y expr=\thisrowno{1} * 10, col sep=comma]{data/optimal/optimalcontrol_w_1.ttt};
%			\addlegendentry{Optimal control}
%			
%			\addplot[
%			color=blue,
%			dashed,
%			]
%			table[x expr=\thisrowno{0},y expr=\thisrowno{1} * 3 + 600, col sep=comma]{data/optimal/optimalcontrol_30PumpSpeed_1.ttt};
%			\addlegendentry{Optimal control setting}
%			
%			\addplot[
%			color=red,
%			]
%			table[x expr=\thisrowno{0},y expr=\thisrowno{1} * 10, col sep=comma]{data/fixed/staticcontrol_w_1.ttt};
%			\addlegendentry{Fixed control}
%			
%			\addplot[
%			color=red,
%			dashed,
%			]
%			table[x expr=\thisrowno{0},y expr=\thisrowno{1} * 3 + 600, col sep=comma]{data/fixed/staticcontrol_30PumpSpeed_1.ttt};
%			\addlegendentry{Fixed control setting}
%			
%			\addplot[
%			color=green,
%			domain=0:120
%			]
%			{850};
%			\addlegendentry{Physical limit}
%		\end{axis}
%		
%		\begin{axis}[
%			scale only axis,
%			ylabel={Rain intensity [my-m/s]},
%			xmin=0, xmax=120,
%			ymin=0, ymax=4,
%			axis y line*=right,
%			axis x line=none,
%			y dir=reverse,
%			width=12cm,
%			height=5cm,
%			%			xtick distance=360,
%			xmajorgrids=true,
%			%			ymajorgrids=true,
%			grid style=dashed,
%			yticklabel style={
%				/pgf/number format/precision=0,
%				/pgf/number format/fixed,
%				/pgf/number format/fixed zerofill,
%			},
%			]
%			
%			\addplot[
%			color=black,
%			]
%			table[x expr=\thisrowno{0},y index = 1, col sep=comma]{data/optimal/optimalcontrol_rain_1.ttt};
%		\end{axis}	
%	\end{tikzpicture}
%\end{frame}
	
\begin{frame}{Experimental results}
	\centering
	\begin{tikzpicture}
		% To plot figure with multiple ordinates, see section 4.9.11 of the pgfplot manual.
		pgfplotsset{set layers}
		\begin{axis}[
			scale only axis,
			xlabel={Time [min]},
			ylabel={Water level [mm]},
			xmin=0, xmax=120,
			ymin=600, ymax=1000,
			axis y line*=left,  %'*' avoids arrow heads
			axis x line*=bottom, %'*' avoids arrow heads
			width=12cm,
			height=5cm,
%			xtick distance=360,
			xmajorgrids=true,
			ymajorgrids=true,
			grid style=dashed,
			yticklabel style={
				/pgf/number format/precision=0,
				/pgf/number format/fixed,
				/pgf/number format/fixed zerofill,
			},
			legend style={
			at={(0.99,0.18)},
			anchor=south east},
			nodes={scale=0.7, transform shape}
			]	
			
			
			\addplot[
			color=blue,
			]
			table[x expr=\thisrowno{0} / 60,y index = 1, col sep=comma]{data/optimal/DepthData.csv};
			\addlegendentry{Optimal control}
						
			\addplot[
			color=blue,
			dashed,
			]
			table[x expr=\thisrowno{0} / 60,y expr=\thisrowno{1} * 30 + 600, col sep=comma]{data/optimal/Control_stateData.csv};
			\addlegendentry{Optimal control setting}
			
			\addplot[
			color=red,
			]
			table[x expr=\thisrowno{0} / 60,y index = 1, col sep=comma]{data/fixed/DepthData.csv};
			\addlegendentry{Fixed control}
			
			\addplot[
			color=red,
			dashed,
			]
			table[x expr=\thisrowno{0} / 60,y expr=\thisrowno{1} * 30 + 600, col sep=comma]{data/fixed/Control_stateData.csv};
			\addlegendentry{Fixed control setting}			
			
			\addplot[
			color=green,
			domain=0:120
			]
			{850};
			\addlegendentry{Physical limit}
		\end{axis}
		
		\begin{axis}[
			scale only axis,
			ylabel={Rain intensity [my-m/s]},
			xmin=0, xmax=120,
			ymin=0, ymax=4,
			axis y line*=right,
			axis x line=none,
			y dir=reverse,
			width=12cm,
			height=5cm,
%			xtick distance=360,
			xmajorgrids=true,
%			ymajorgrids=true,
			grid style=dashed,
			yticklabel style={
				/pgf/number format/precision=0,
				/pgf/number format/fixed,
				/pgf/number format/fixed zerofill,
			},
			]
			
			\addplot[
			color=black,
			]
			table[x expr=\thisrowno{0} / 60,y index = 1, col sep=comma]{data/optimal/RainData.csv};
		\end{axis}	
	\end{tikzpicture}
\end{frame}

\begin{frame}{Analysis}
	\begin{block}{Possible explanations for results}
		\begin{itemize}
			\item Learning took about 1 minute, so plenty of computational budget left.
			\item Switching to 20 minutes control period reduced the number of possible control sequences to analyze, hence more likely to observe a good control strategy during learning.
			\item Overflow is avoided much longer compared to static control.
			\item Once the water level gets close to maximum, it seems hard for learning to get out of this state.
		\end{itemize}
	\end{block}
	
	\begin{block}{Idea for next experiment}
		\begin{itemize}
			\item Not only penalizing for overflow duration, but overflow amount. This seems to give better results when I run offline control on my local laptop.
		\end{itemize}
	\end{block}
\end{frame}


\end{document}
